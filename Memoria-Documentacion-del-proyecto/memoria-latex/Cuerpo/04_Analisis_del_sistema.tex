\section{Análisis del sistema}
\label{sec:analisis}

\subsection{Descripción general del sistema}
\label{subsec:descripcion-general}

El sistema desarrollado tiene como objetivo servir de herramienta de apoyo para la gestión de un almacén de consolidación en un entorno industrial. La aplicación permite centralizar la información relacionada con productos, palets, ubicaciones, movimientos y pedidos, facilitando la operativa diaria del almacén y el seguimiento de la mercancía a lo largo de los distintos procesos internos.

La solución se concibe como un sistema distribuido que da soporte a distintos perfiles de usuario mediante interfaces diferenciadas. Por un lado, una aplicación de escritorio orientada a tareas de gestión, planificación y supervisión; por otro, una aplicación móvil destinada a la operativa en planta, permitiendo el acceso a la información y el registro de acciones directamente en el entorno de trabajo.

El sistema está específicamente diseñado para almacenes de consolidación, donde se agrupan mercancías procedentes de distintos proveedores con el fin de preparar envíos conjuntos a clientes. En consecuencia, el alcance funcional se centra en procesos como la gestión de inventario, la consolidación de pedidos y la trazabilidad de los movimientos internos, sin abordar otros tipos de almacén ni el control directo de sistemas automatizados.


\subsection{Identificación de usuarios}
\label{subsec:usuarios}

\begin{table}[H]
\centering
\small
\begin{tabular}{p{6.2cm}cccc}
\toprule
\textbf{Funcionalidad} & \textbf{SysAdmin} & \textbf{Gestor Almacén} & \textbf{Operario} & \textbf{Administración} \\
\midrule
Acceso a la gestión general del almacén & \permits & \permits & \permits & \denies \\
Consulta y gestión del inventario & \permits & \permits & \permits & \denies \\
Gestión de pedidos & \permits & \permits & \denies & \permits \\
Paletización de mercancía & \permits & \permits & \permits & \denies \\
Gestión de envíos & \permits & \permits & \permits & \denies \\
Consulta del calendario de operaciones & \permits & \permits & \permits & \permits \\
Registro de movimientos de mercancía & \permits & \permits & \permits & \denies \\
Actualización de información de palets & \permits & \permits & \denies & \denies \\
Gestión de órdenes de compra & \permits & \permits & \denies & \permits \\
Exportación de datos del sistema & \permits & \denies & \denies & \permits \\
Creación de pedidos & \permits & \permits & \denies & \permits \\
Edición de pedidos & \permits & \permits & \denies & \permits \\
Eliminación de pedidos & \permits & \permits & \denies & \permits \\
Creación de productos & \permits & \permits & \denies & \denies \\
Creación de tipos de producto & \permits & \permits & \denies & \denies \\
Creación de usuarios & \permits & \denies & \denies & \permits \\
Edición de usuarios & \permits & \denies & \denies & \permits \\
Eliminación de usuarios & \permits & \denies & \denies & \permits \\
\bottomrule
\end{tabular}
\caption{Permisos de acceso a funcionalidades del sistema según rol de usuario.}
\label{tab:permisos-funcionalidades-rol}
\end{table}

El sistema contempla distintos tipos de usuarios o roles, definidos en función de su rol dentro del almacén. La aplicación de los roles se aplica exclusivamente a la aplicación de escritorio, donde se concentran las funcionalidades de gestión, administración y supervisión del sistema. En la aplicación móvil Android, orientada a la operativa en planta, todos los usuarios actúan bajo un perfil funcional equivalente al de operario, independientemente de su rol dentro de la organización.

En la aplicación de escritorio se han definido cuatro roles principales: Administrador del sistema (SysAdmin), Gestor de almacén, Operario y Administración. Cada uno de estos roles dispone de un conjunto específico de permisos que determinan el acceso a las distintas funcionalidades del sistema. La Tabla~\ref{tab:permisos-funcionalidades-rol} recoge de forma resumida los permisos asignados a cada rol en relación con las funcionalidades disponibles. Por ejemplo: Un SysAdmin puede crear, editar y eliminar usuarios, mientras que un Operario no tiene acceso a estas funciones.

El rol de SysAdmin dispone de acceso completo a todas las funcionalidades del sistema, incluyendo la gestión de usuarios, la configuración general, la administración de productos, pedidos y la exportación de datos. Este perfil está orientado a tareas de administración avanzada y mantenimiento del sistema.

El Gestor de almacén es responsable de la planificación y supervisión de la operativa diaria. Este rol puede gestionar inventario, pedidos, palets, envíos y órdenes de compra, así como realizar tareas relacionadas con la organización del almacén, pero no dispone de permisos para la gestión de usuarios ni para la exportación de datos del sistema.

El rol de Administración está orientado a tareas de carácter administrativo y documental. Este perfil puede gestionar pedidos, órdenes de compra, usuarios y exportar información del sistema, pero no interviene directamente en la operativa física del almacén ni en la gestión de movimientos de mercancía.

Por último, el Operario de almacén se encarga de ejecutar las tareas operativas relacionadas con la manipulación y movimiento de mercancías. En la aplicación de escritorio, su acceso se limita a funcionalidades operativas como la consulta de información, la paletización, el registro de movimientos y la gestión de envíos, sin permisos para tareas de administración o gestión avanzada. En la aplicación móvil Android, este rol constituye el único perfil de uso, permitiendo registrar movimientos, consultar inventario y apoyar los procesos de consolidación directamente en planta.

Esta estructura de roles y permisos permite limitar las funcionalidades y las interfaces del sistema a las responsabilidades de cada tipo de usuario, mejorando la eficiencia operativa, reforzando el control de acceso y reduciendo la probabilidad de errores derivados del uso indebido de funcionalidades no autorizadas.


\subsection{Requisitos del sistema}
\label{subsec:requisitos}


\subsubsection{Requisitos funcionales}
\label{subsubsec:requisitos-funcionales}

El sistema debe cumplir los siguientes requisitos funcionales:

\begin{itemize}
    \item Permitir la gestión de productos y sus características asociadas.
    \item Gestionar palets y su ubicación dentro del almacén.
    \item Registrar movimientos internos de mercancía y cambios de ubicación.
    \item Facilitar la preparación y consolidación de pedidos.
    \item Permitir el acceso al sistema a distintos perfiles de usuario.
    \item Proporcionar interfaces diferenciadas para tareas de gestión y operativa en planta.
    \item Consultar el estado del inventario y la trazabilidad de los productos.
\end{itemize}


\subsubsection{Requisitos no funcionales}
\label{subsubsec:requisitos-no-funcionales}

Además de los requisitos funcionales, el sistema debe cumplir una serie de requisitos no funcionales:

\begin{itemize}
    \item El sistema debe ser fácil de usar por personal con distintos niveles de experiencia.
    \item La información debe mantenerse consistente y actualizada en todo momento.
    \item El acceso a los datos debe estar controlado según el perfil de usuario.
    \item El sistema debe ser mantenible y permitir futuras ampliaciones funcionales.
    \item La solución debe ser compatible con entornos industriales habituales.
    \item El sistema debe garantizar la seguridad de las credenciales de los usuarios, evitando el almacenamiento de contraseñas en texto plano y aplicando mecanismos de protección adecuados frente a accesos no autorizados.

\end{itemize}


\subsection{Restricciones técnicas}
\label{subsec:restricciones}

El desarrollo del sistema ha estado condicionado por una serie de restricciones técnicas y de alcance propias de un Trabajo Fin de Grado.

En primer lugar, la aplicación no contempla el control directo de maquinaria ni de sistemas de automatización industrial, centrándose exclusivamente en la gestión de la información asociada al almacén. Asimismo, el sistema ha sido diseñado específicamente para un entorno de almacén de consolidación, por lo que no resulta aplicable de forma directa a otros tipos de almacén con procesos logísticos diferentes.

Por último, se han considerado las limitaciones temporales y de recursos inherentes a un Trabajo Fin de Grado, desarrollado por un único autor y dentro de un periodo de tiempo acotado. Como consecuencia, se ha priorizado la implementación de las funcionalidades esenciales para la gestión del almacén de consolidación, quedando fuera del alcance del proyecto aspectos como la integración con sistemas externos, el control directo de equipos automatizados o el desarrollo de funcionalidades avanzadas de optimización.



