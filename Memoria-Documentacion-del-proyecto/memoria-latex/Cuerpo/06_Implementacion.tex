\section{Implementación}
\label{sec:implementacion}

\subsection{Herramientas y tecnologías utilizadas}
\label{subsec:herramientas}

El sistema desarrollado se ha implementado íntegramente utilizando el lenguaje de programación Java. Este lenguaje se ha empleado tanto en el desarrollo de la aplicación de escritorio como en la aplicación móvil Android y en el backend del sistema, implementado mediante el framework Spring Boot.

La utilización de un lenguaje común en todos los componentes de la solución facilita la integración entre las distintas aplicaciones, simplifica el mantenimiento del sistema y permite reutilizar conocimientos y conceptos a lo largo de todo el desarrollo. Además, Java es un lenguaje ampliamente utilizado en entornos industriales y empresariales, caracterizado por su portabilidad, robustez y amplio ecosistema de herramientas y librerías \cite{oracle_java}.


\begin{itemize}
    \item \textbf{Control de versiones}: Git.
    \item \textbf{Backend}: Java junto con el framework Spring Boot.
    \item \textbf{Base de datos}: MySQL.
    \item \textbf{Gestión y modelado de datos}: MySQL Workbench y Draw.io.
    \item \textbf{Desarrollo de la aplicación de escritorio}: IntelliJ IDEA Ultimate.
    \item \textbf{Desarrollo de la aplicación móvil}: Android Studio.
    \item \textbf{Pruebas de la API}: Postman.
    \item \textbf{Entorno de despliegue}: VMware Workstation con un servidor Ubuntu.
    \item \textbf{Gestión del proyecto}: Git.
\end{itemize}


\subsection{Implementación de la aplicación Desktop}
\label{subsec:impl-desktop}
% TODO: Estructura del proyecto, módulos, pantallas principales, etc.

\subsection{Implementación de la aplicación Android}
\label{subsec:impl-android}
% TODO: Componentes, pantallas, consumo API, gestión de estado, etc.

\subsection{Implementación del backend / servidor}
\label{subsec:impl-backend}
% TODO: Controladores, servicios, repositorios, validaciones, etc.

\subsection{Implementación de la base de datos}
\label{subsec:impl-bd}
% TODO: Scripts, migraciones, constraints, índices, datos de prueba.

\subsection{Seguridad y gestión de errores}
\label{subsec:seguridad-errores}

En el diseño del sistema se ha tenido en cuenta la protección de las credenciales de los usuarios. Las contraseñas no se almacenan en texto claro en la base de datos, sino que se guardan en forma de valores hash generados mediante un algoritmo de derivación de claves seguro.

Para este propósito se ha optado por el uso de Argon2, un algoritmo específicamente diseñado para el almacenamiento seguro de contraseñas, que ofrece resistencia frente a ataques de fuerza bruta y ataques mediante hardware especializado.


\subsection{Control de versiones y gestión del proyecto}
\label{subsec:control-versiones}
% TODO: Git, ramas, issues, milestones, metodología (Scrum/Kanban), etc.
