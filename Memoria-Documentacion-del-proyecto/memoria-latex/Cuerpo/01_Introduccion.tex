\section{Introducción}
\label{sec:introduccion}

La optimización de los procesos industriales y logísticos constituye uno de los pilares fundamentales en los entornos productivos actuales. Aspectos como el control operativo, la trazabilidad de los materiales o la reducción de errores influyen de manera directa tanto en los costes como en la calidad del servicio ofrecido. Dentro de este escenario, el almacén adquiere un papel determinante en la cadena de suministro, al actuar como punto de conexión entre producción, transporte y distribución.

Entre los distintos modelos de almacenamiento existentes, los almacenes de consolidación desempeñan una función específica centrada en la agrupación, organización y preparación de mercancía procedente de múltiples orígenes para su expedición conjunta. Este tipo de instalaciones presenta retos operativos propios, entre los que destacan un elevado número de movimientos, una rotación continua de productos y la necesidad de mantener un seguimiento preciso de palets, ubicaciones y pedidos.


\begin{figure}[H]
    \centering
    \includegraphics[width=0.5\textwidth]{Imagenes/almacen-consolidacion.png}
    \caption{Almacén de consolidación en entorno industrial}
    \label{fig:almacen-consolidacion}
\end{figure}

La incorporación de sistemas digitales en la gestión de estos almacenes permite un mayor control del inventario, una localización más precisa de los materiales y una coordinación más ordenada de los flujos de entrada y salida. En este contexto se enmarca el presente Trabajo Fin de Grado, cuyo propósito es el desarrollo de una solución técnica orientada de forma específica a la gestión de almacenes de consolidación, concebida como apoyo directo a la operativa industrial diaria.

El proyecto aborda el diseño e implementación de un sistema compuesto por una aplicación de escritorio, una aplicación móvil para dispositivos Android y un backend responsable del tratamiento y almacenamiento de la información. El planteamiento adoptado responde a un enfoque práctico, alineado con los conocimientos adquiridos en la titulación de Ingeniería en Electrónica Industrial y Automática, y orientado a la aplicación real de tecnologías software como soporte a procesos logísticos.

\subsection{Contexto y motivación del proyecto}
\label{subsec:contexto-motivacion}

En entornos industriales y logísticos, la gestión del almacén condiciona de forma directa la continuidad de la producción y el cumplimiento de los plazos establecidos. En los almacenes de consolidación, una gestión deficiente puede derivar en desajustes de inventario, tiempos improductivos durante la preparación de pedidos, errores en la agrupación de mercancía o dificultades para conocer el estado real de los materiales.

Existen soluciones comerciales destinadas a la gestión de almacenes, aunque muchas de ellas están orientadas a grandes centros logísticos generalistas o incorporan un nivel de complejidad elevado. Esta situación complica su implantación en instalaciones de menor escala o con flujos de trabajo muy concretos, donde la inversión en infraestructura, formación o consultoría resulta poco viable.

La motivación principal de este proyecto surge de la necesidad de contar con una herramienta accesible y comprensible, pensada para entornos que carecen de soluciones específicas para la gestión de almacenes de consolidación. La propuesta se centra en cubrir las funciones esenciales de este tipo de instalaciones e integrarse de forma natural en la rutina diaria del almacén mediante interfaces claras y orientadas al usuario. La combinación de una aplicación de escritorio para tareas de planificación y una aplicación móvil para la operativa en planta responde a situaciones habituales en entornos industriales reales.

Desde el punto de vista académico, el desarrollo del proyecto permite aplicar conocimientos relacionados con sistemas de información, arquitectura de software, gestión de datos y diseño de interfaces, reforzando la formación técnica en automatización y electrónica industrial con una visión orientada a la resolución de problemas logísticos.

\subsection{Problema a resolver}
\label{subsec:problema}

El problema abordado en este Trabajo Fin de Grado se centra en el desarrollo de sistemas integrados y adaptados a la realidad de los almacenes de consolidación en entornos industriales. En concreto, se plantea la necesidad de gestionar de manera estructurada la información asociada a productos, palets, ubicaciones, movimientos y pedidos.

En numerosas instalaciones, estas tareas se apoyan en herramientas genéricas, soluciones parciales o procedimientos manuales, lo que incrementa el riesgo de errores y dificulta el acceso a información actualizada y coherente. Este escenario puede traducirse en pérdidas de material, desajustes de inventario y una trazabilidad limitada, especialmente en instalaciones con alta rotación de mercancía.

El sistema propuesto plantea una respuesta a esta problemática mediante una solución que:
\begin{itemize}
    \item Centraliza la información del almacén de consolidación en una base de datos común.
    \item Permite el control estructurado del inventario y la gestión de ubicaciones.
    \item Apoya la gestión de pedidos y el registro de movimientos durante los procesos de consolidación.
\end{itemize}

El alcance del proyecto se limita al desarrollo de un sistema de apoyo a la gestión del almacén, sin intervenir en el control directo de equipos automatizados. Aun así, la arquitectura planteada contempla la posibilidad de integración futura con sistemas de control de planta o automatización industrial.

\subsection{Objetivos del TFG}
\label{subsec:objetivos}

El objetivo general de este Trabajo Fin de Grado consiste en el desarrollo de un sistema de gestión orientado a almacenes de consolidación en entornos industriales, con el fin de mejorar el control, la organización y la trazabilidad de los materiales mediante aplicaciones software integradas.

A partir de este objetivo general, se definen los siguientes objetivos específicos:
\begin{itemize}
    \item Estudiar las necesidades operativas propias de un almacén de consolidación industrial.
    \item Definir una arquitectura que permita la comunicación entre aplicaciones de escritorio, móviles y un backend común.
    \item Desarrollar la gestión de productos, palets, ubicaciones y pedidos.
    \item Implementar un sistema de almacenamiento de datos que asegure la coherencia y disponibilidad de la información.
    \item Diseñar interfaces orientadas a la operativa diaria del personal del almacén.
\end{itemize}
