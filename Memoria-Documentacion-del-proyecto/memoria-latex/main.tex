\documentclass[11pt]{article}
%\documentclass[11pt, twoside]{article}

% --- Codificación y fuentes ---
\usepackage[T1]{fontenc}
\usepackage{mathptmx}
\usepackage[spanish, es-tabla]{babel}
\usepackage{kpfonts}

% --- Paquetes generales ---
\usepackage{graphicx}
\usepackage[table,xcdraw]{xcolor}
\usepackage{xcolor}
\usepackage{array}
\usepackage{enumitem}
\usepackage{ragged2e}
\usepackage{amsmath}
\usepackage{float}
\usepackage{booktabs}
\usepackage{svg}
\usepackage{fontspec}
\usepackage{booktabs}
\usepackage{fancyvrb}
\usepackage[acronym]{glossaries}
\makeglossaries

% --- Fuentes ---
% \setmonofont[
%     Path = "fonts/FiraCode/",
%     Extension = .ttf,
%     UprightFont = *-Regular,
%     BoldFont = *-Bold,
% ]{FiraCode}
\setmonofont{FiraCode Nerd Font Mono}


% --- Márgenes y geometría ---
\usepackage{geometry}
\geometry{
  a4paper,
  inner=30mm,    % margen interior (encuadernación)
  outer=15mm,    % margen exterior
  top=25mm,
  bottom=25mm,
  headheight=45pt,
  headsep=8pt
}

\usepackage[utf8]{inputenc}

% --- Encabezado y pie de página ---
\usepackage{fancyhdr}

% --- Estilo secciones ---
\definecolor{AzulTFG}{RGB}{0, 148, 224}
\usepackage{titlesec}

% --- Interlineado entre párrafos y sangría ---
\usepackage{parskip}

% --- Captions sin negrita ---
\usepackage{caption}
\usepackage{subcaption}
\captionsetup{labelfont=normalfont}

% --- Hipervínculos ---
\usepackage[hidelinks]{hyperref}

% --- Formato índice ---
\usepackage[nottoc]{tocbibind}
% (Esta línea en tu fichero original parece pensada para gallego,
%  pero la mantengo por compatibilidad. En español no hace efecto.)
\addto\captionsgalician{\renewcommand{\contentsname}{Contenido}}
\usepackage{tocloft}

% --- Bibliografía ---
\addto\captionsspanish{%
  \renewcommand{\refname}{Bibliografía}
}

\usepackage[
  backend=biber,
  style=ieee,
  sorting=none,
  autolang=other,
  bibencoding=utf8,
  url=true,
  doi=true,
  isbn=true
]{biblatex}

\addbibresource{Bibliografia/bibliografia.bib}

% --- ANEXOS ---

\usepackage{listings}

\lstdefinelanguage{SQL}{
  keywords={CREATE, TABLE, PRIMARY, KEY, FOREIGN, REFERENCES, NOT, NULL, UNIQUE, INSERT, INTO, VALUES, ALTER, ADD, CONSTRAINT, ON, DELETE, UPDATE, VARCHAR, INT, TIMESTAMP, DEFAULT, AUTO_INCREMENT, BOOLEAN, DELIMITER, NEW, SELECT, FROM, WHERE, AND, OR, DROP, DATABASE, IF, EXISTS, SET, END, CONCAT, DATE_FORMAT, ROW, BEGIN, DECLARE, THEN, IS},
  sensitive=false,
  morecomment=[l]{--},
  morestring=[b]'
}

\lstset{
  basicstyle=\ttfamily\small,
  numbers=left,          % números a la izquierda
  numberstyle=\tiny,     % tamaño del número
  stepnumber=1,          % numerar todas las líneas
  numbersep=8pt,         % separación número-código
  xleftmargin=1.8em,
  framexleftmargin=1.8em,
  breaklines=true,
  frame=single,
  keywordstyle=\color{blue}\bfseries,
  commentstyle=\color{gray},
  stringstyle=\color{green!40!black},
  columns=fullflexible,
  inputencoding=utf8,
  extendedchars=true,
  literate=
  {á}{{\'a}}1
  {é}{{\'e}}1
  {í}{{\'i}}1
  {ó}{{\'o}}1
  {ú}{{\'u}}1
  {Á}{{\'A}}1
  {É}{{\'E}}1
  {Í}{{\'I}}1
  {Ó}{{\'O}}1
  {Ú}{{\'U}}1
  {ñ}{{\~n}}1
  {Ñ}{{\~N}}1
  {-}{{\~-}}1
}


% Título de la bibliografía (equivalente a \refname)
\DefineBibliographyStrings{spanish}{%
  bibliography = {Bibliografía},
}
\newcommand{\footcitepage}[3]{%
  ``#1''\footnote{%
    \fullcite{#2}, p.~#3%
  }%
}

\newcommand{\footbibliographypage}[2]{%
  \footnote{%
    \fullcite{#1}, p.~#2%
  }%
}

\newcommand{\footwebfragment}[2]{%
  \footnote{%
    \fullcite{#1}. Fragmento: #2%
  }%
}

\newcommand{\footweb}[1]{%
  \footnote{%
    \fullcite{#1}
  }%
}


\newcommand{\footwebcite}[3]{%
  ``#1''\footnote{%
    \fullcite{#2}. Fragmento: #3%
  }%
}

\newcommand{\footcitedefinition}[3]{%
  #1\footnote{%
    \fullcite{#2}, p.~#3%
  }%
}

% --- Glosario ---
\renewcommand*{\glossaryentrynumbers}[1]{}

% --- Portada ---
\usepackage{framed}
\newcommand{\printportada}[6]{%
  \newgeometry{
    left=30mm,
    right=15mm,
    top=25mm,
    bottom=0mm,
    headheight=31.1pt
  }%
  \setlength{\headsep}{-35pt}%

  % ===== Portada 1 =====
  \begin{titlepage}
    \thispagestyle{empty}
    \begin{framed}
      \begin{center}
        \includegraphics[scale=0.7]{FotosPlantilla/EEI - logo2.png}

        \vspace{-5pt}
        \fontsize{24pt}{12}\selectfont{\color{AzulTFG}Escuela de Ingeniería Industrial}

        \vspace{2.3cm}
        \fontsize{17.5pt}{10}\textbf{\scshape{Trabajo Fin de Grado}}

        \vspace{1.7cm}
        \fontsize{18.5pt}{10}\normalfont{\itshape{#1}}

        \vspace{1.7cm}
        \fontsize{14.5pt}{10}\textbf{Grado en Ingeniería en #2}

        \vspace{3cm}
        \begin{tabular}{l l}
          {\hspace{0cm}}\fontsize{14pt}{10}\textbf{\scshape{Alumno:}} &
          \fontsize{14.5pt}{10}\normalfont{#3} \\ \\
          {\hspace{0cm}}\fontsize{14pt}{10}\textbf{\scshape{Directores:}} &
          \fontsize{14.5pt}{10}\normalfont{#4} \\ \\
          & \fontsize{14.5pt}{10}\normalfont{#5}
        \end{tabular}

        \vspace{3cm}
        \includegraphics[scale=1]{FotosPlantilla/UVIGO - Logo.png}
      \end{center}
    \end{framed}
  \end{titlepage}

  % ===== Portada 2 =====
  \begin{titlepage}
    \thispagestyle{empty}
    \begin{framed}
      \begin{center}
        \includegraphics[scale=0.7]{FotosPlantilla/EEI - logo2.png}

        \vspace{-5pt}
        \fontsize{24pt}{12}\selectfont{\color{AzulTFG}Escuela de Ingeniería Industrial}

        \vspace{2.3cm}
        \fontsize{17.5pt}{10}\textbf{\scshape{Trabajo Fin de Grado}}

        \vspace{1.7cm}
        \fontsize{18.5pt}{10}\normalfont{\itshape{#1}}

        \vspace{1.7cm}
        \fontsize{14.5pt}{10}\textbf{Grado en Ingeniería en #2}

        \vspace{3cm}
        \fontsize{14.5pt}{10}\textbf{Documento}

        \vspace{1.1cm}
        \fontsize{17.5pt}{10}\textbf{\scshape{#6}}

        \vspace{3cm}
        \includegraphics[scale=1]{FotosPlantilla/UVIGO - Logo.png}
      \end{center}
    \end{framed}
  \end{titlepage}

  \restoregeometry
  \setcounter{page}{1}%
}



% \pagestyle{fancy}
% \fancyhf{}

% % Página impar (derecha)
% \fancyhead[RO]{\thepage}
% \fancyhead[LO]{\nouppercase{\rightmark}}

% % Página par (izquierda)
% \fancyhead[LE]{\thepage}
% \fancyhead[RE]{\nouppercase{\leftmark}}

% \renewcommand{\headrulewidth}{0.4pt}


\newcommand{\permits}{\textcolor{green!60!black}{Sí}}
\newcommand{\denies}{\textcolor{red!70!black}{No}}

\begin{document}

% --- Portada ---
% Cambia los parámetros según tu caso
\printportada{Diseño de una aplicación informática para la gestión de un almacén sin automatización con distintas interfaces para los operarios}{Electrónica Industrial y Automática}{José Tomás Torre Pedroarena}{Joaquín lópez Fernández}{}{Memoria}

% --- Índices ---

\newgeometry{
 left=30mm,
 right=15mm,
 top=25mm,
 bottom=36mm,
 headheight=31.1pt
 }
 \setlength{\headsep}{35pt}
 %Cabecera y pie de página
\tocloftpagestyle{fancy}
\pagestyle{fancy}
\fancyhf{}
\fancyhead[LH]{\fontsize{14pt}{12} \selectfont\textsc{Diseño de una aplicación informática para la gestión de un almacén sin automatización con distintas interfaces para los operarios} \\[6pt]  \textsc{José Tomás Torre Pedroarena}}
\cfoot{\thepage}

%formato contido
\renewcommand{\cftsecfont}{\normalfont}
\renewcommand{\cftsecdotsep}{1}
\renewcommand{\cftdotsep}{1}
\renewcommand{\cftsecpagefont}{\normalfont}
\renewcommand{\cfttoctitlefont}{\fontsize{17.5pt}{0}\scshape\bfseries\color{AzulTFG}}
\renewcommand{\cftdot}{\normalfont.}
\setlength{\cftsecindent}{2em}
\setlength{\cftsubsecindent}{3em}
\setlength{\cftsubsubsecindent}{4em}
\renewcommand{\numberline}[1]{#1 \hspace{1pt}}

%formato lista figuras
\renewcommand{\cftloftitlefont}{\fontsize{17.5pt}{0}\scshape\bfseries\color{AzulTFG}}

%formato lista de tablas
\renewcommand{\cftlottitlefont}{\fontsize{17.5pt}{0}\scshape\bfseries\color{AzulTFG}}


\tableofcontents
\clearpage


% estilo de títulos (titlesec)
\restoregeometry
\setlength{\headsep}{4pt}
%Cabecera y pie de página
\pagestyle{fancy}
\fancyhf{}

\fancyhfoffset[L]{0pt} % sin desplazamiento a la izquierda
\fancyhfoffset[R]{0pt} % sin desplazamiento a la derecha
\setlength{\headwidth}{\textwidth} % que la cabecera use el ancho del texto


\fancyhead[L]{\fontsize{14pt}{12} \selectfont\textsc{Diseño de una aplicación informática para la gestión de un almacén sin automatización con distintas interfaces para los operarios} \\[6pt]  \textsc{José Tomás Torre Pedroarena}}
\cfoot{\thepage}

%Estilo seccion
\titleformat{\section}{\normalfont\fontsize{17.5pt}{20}\scshape\bfseries\color{AzulTFG}}{\thesection}{0.3em}{}
\titlespacing\section{0pt}{12pt}{10pt plus 2pt minus 2pt}

%Estilo subsección
\titleformat{\subsection}{\normalfont\fontsize{15.5pt}{20}\bfseries}{\thesubsection}{0.3em}{}
\titlespacing\subsection{0pt}{12pt plus 4pt minus 2pt}{0pt plus 2pt minus 2pt}

%Estilo subsubsección
\titlespacing\subsubsection{0pt}{12pt plus 4pt minus 2pt}{0pt plus 2pt minus 2pt}
\titleformat{\subsubsection}{\normalfont\fontsize{15.5pt}{20}\itshape}{\thesubsubsection}{0.3em}{}

%Espacio debaijo de captions
\setlength{\belowcaptionskip}{0pt}
%Espacio arriba de captions
\setlength{\abovecaptionskip}{2pt}


%interlineado entre parágrafo y sangrado
\setlength{\parskip}{4pt} %interlineado
\setlength{\parindent}{20pt} %sangrado



\glsaddall
\vspace{-1cm}
% =========================
% Glosario de términos
% =========================

\newglossaryentry{backend}{
    name=Backend,
    description={Conjunto de servidores y componentes software encargados de proporcionar servicios, gestionar la lógica de negocio y permitir el acceso a recursos y datos desde aplicaciones cliente}
}

\newglossaryentry{framework}{
    name=Framework,
    description={Estructura de desarrollo compuesta por componentes de software reutilizables, librerías y convenciones que facilitan la creación de aplicaciones de forma estandarizada y eficiente}
}

\newglossaryentry{sku}{
    name=SKU,
    description={Stock Keeping Unit, identificador único de un producto en el inventario}
}



% =========================
% Glosario de siglas
% =========================

\newacronym
  [plural=WMS,
   longplural={Sistemas de Gestión de Almacenes}]
  {wms}{WMS}{Sistema de Gestión de Almacenes}

\newacronym{api}{API}{Interfaz de Programación de Aplicaciones}
\newacronym{rest}{REST}{Transferencia de Estado Representacional}
\newacronym{erp}{ERP}{Planificación de Recursos Empresariales}
\newacronym{gui}{GUI}{Interfaz Gráfica de Usuario}
\newacronym{sql}{SQL}{Lenguaje de Consulta Estructurado}
\newacronym{http}{HTTP}{Hypertext Transfer Protocol}
\newacronym{https}{HTTPS}{Hypertext Transfer Protocol Secure}
\newacronym{json}{JSON}{JavaScript Object Notation}
\newacronym{xml}{XML}{Extensible Markup Language}

% =========================
% Impresión del glosario
% =========================

\section*{Glosario de siglas}
\label{sec:glosario-siglas}
\vspace{-1.3cm}

\printglossary[type=\acronymtype, title={}, toctitle={}]

\section*{Definiciones}
\label{sec:glosario-terminos}
\vspace{-1.3cm}

\printglossary[title={}, toctitle={}]

\clearpage

% =========================================================
% CUERPO DEL TRABAJO (estructura solicitada)
% =========================================================



% 1 Introducción
\section{Introducción}
\label{sec:introduccion}

\subsection{Contexto y motivación del proyecto}
\label{subsec:contexto-motivacion}
% TODO: Escribe aquí el contexto, la motivación y por qué es relevante.

\subsection{Problema a resolver}
\label{subsec:problema}
% TODO: Describe el problema concreto, alcance y por qué hay que resolverlo.

\subsection{Objetivos del TFG}
\label{subsec:objetivos}

Según Pressman \cite{pressman_software_engineering}, la ingeniería del software

% TODO: Enumera objetivos generales y específicos. Puedes usar itemize.
% \begin{itemize}
%   \item Objetivo general...
%   \item Objetivo específico...
% \end{itemize}


% 2 Estado del arte
\section{Estado del arte}
\label{sec:estado-arte}

\subsection{Sistemas de gestión de almacenes (WMS)}
\label{subsec:wms}
% TODO: Explica qué es un WMS, propósito, componentes, etc.

\subsection{Funcionalidades habituales en aplicaciones de gestión de almacenes}
\label{subsec:funcionalidades-wms}
% TODO: Recepción, ubicación, picking, inventario, expedición, trazabilidad, etc.

\subsection{Aplicaciones comerciales de gestión de almacenes}
\label{subsec:aplicaciones-comerciales}
% TODO: Presenta ejemplos de soluciones comerciales y características comparables.

\subsection{Justificación de la solución propuesta}
\label{subsec:justificacion}
% TODO: Conecta el estado del arte con tu propuesta: por qué tu solución tiene sentido.


% 3 Marco teórico y tecnológico
\section{Marco teórico y tecnológico}
\label{sec:marco-teorico}

En este capítulo se presentan los fundamentos teóricos y tecnológicos que se emplearon en el desarrollo del sistema propuesto. Se abordan los conceptos relacionados con los sistemas de gestión de almacenes, las arquitecturas cliente-servidor, el desarrollo de aplicaciones multiplataforma, las aplicaciones móviles Android, las aplicaciones de escritorio y los mecanismos de comunicación entre aplicaciones, incluyendo el uso de \glspl{api} de tipo \gls{rest} en determinados componentes del sistema.

\subsection{Aplicaciones multiplataforma}
\label{subsec:multiplataforma}

¿Cómo logramos que el jefe de almacén y el operario empleen el mismo sistema e interactúen con los mismos datos? La respuesta está en el desarrollo multiplataforma. En un entorno donde conviven ordenadores de oficina y terminales móviles, es vital que la lógica del sistema no se fragmente. Al compartir una base común (base de datos) ambos dispositivos deben interactuar entre sí y con el servidor de manera coherente.

Desde un punto de vista funcional, la adopción de soluciones multiplataforma facilita la reutilización de la lógica de negocio y la centralización de la información, reduciendo los costes de mantenimiento y mejorando la coherencia del sistema. En el ámbito de la gestión de almacenes, este enfoque permite que distintos perfiles de usuario accedan al sistema desde interfaces adaptadas a sus necesidades manteniendo una única fuente de datos consistente, por ejemplo: un operario necesita una dispositivo movil para poder realizar sus actividades, por tanto, usará un dispositivo móvil mientras que el jefe de almacén, que requiere una interfaz más completa, accederá desde un ordenador de oficina.

No obstante, el desarrollo multiplataforma también plantea retos, como la adaptación de la interfaz de usuario a diferentes dispositivos o la necesidad de garantizar un comportamiento homogéneo, es decir, que las operaciones realizadas desde un dispositivo se reflejen correctamente en el otro. Estos aspectos deben ser considerados cuidadosamente durante el diseño y la implementación del sistema.


\subsection{Arquitecturas cliente-servidor}
\label{subsec:cliente-servidor}

\begin{figure}[H]
    \centering
    \includegraphics[width=0.7\textwidth]{Imagenes/cliente-servidor.png}
    \caption{Arquitectura cliente-servidor}
    \label{fig:cliente-servidor}
\end{figure}


La arquitectura cliente-servidor es un modelo ampliamente utilizado en \footcitedefinition{sistemas distribuidos}{uvigo_cliente_servidor}{1}, en el que las funciones del sistema se reparten entre componentes cliente y un servidor central. Los clientes se encargan de la interacción con el usuario, mientras que el servidor gestiona la lógica de negocio y la persistencia de los datos.

Este modelo resulta adecuado y conveniente para sistemas de gestión en entornos industriales, dado que permite centralizar la información y garantizar el acceso concurrente desde múltiples dispositivos. El sistema desarrollado se despliega en una red interna privada, prescindiendo del acceso a Internet, donde tanto la aplicación de escritorio como las dispositivos móviles Android funcionan como clientes que se comunican con el servidor central.

Además, este tipo de arquitectura favorece la escalabilidad del sistema y su posible integración futura con otros sistemas, como plataformas empresariales o sistemas de automatización industrial, al proporcionar un punto centralizado de acceso a los servicios y a los datos.



\subsection{Aplicaciones móviles Android}
\label{subsec:android}

Android es uno de los sistemas operativos más extendidos en dispositivos móviles,  \footwebcite{un 78,8\,\% de los españoles con smartphone tuvieron Android como sistema operativo}{cnmc_android_uso_2023}{gráfica ``Sistema operativo del smartphone''}. 

Según la documentación oficial de Android este es una arquitectura basada en componentes como actividades,  servicios, entre otros, que permiten gestionar la interfaz de usuario y el ciclo de vida de la aplicación \cite{android_developers_architectures}.

En entornos industriales y logísticos, el uso de aplicaciones móviles facilita el acceso a la información en tiempo real y reduce la necesidad de desplazamientos innecesarios o registros manuales. En particular, en almacenes de consolidación, las aplicaciones Android permiten registrar movimientos, consultar inventario o identificar palets directamente en el punto de operación.

Dentro del sistema desarrollado, la aplicación android pretende ser una herramienta de apoyo para los operarios del almacén, proporcionando una interfaz sencilla para el registro de las operaciones que estos realizan en planta.

\subsection{Aplicaciones de escritorio}
\label{subsec:desktop}

Las aplicaciones de escritorio continúan desempeñando un papel fundamental en entornos industriales para tareas de gestión, supervisión y planificación. Estas aplicaciones suelen ejecutarse en estaciones de trabajo fijas y permiten ofrecer interfaces más completas para la visualización y administración de grandes volúmenes de información.

En el sistema desarrollado, la aplicación de escritorio se concibe como la herramienta principal para la gestión del almacén de consolidación, permitiendo la administración de productos, ubicaciones, pedidos y usuarios. Este tipo de aplicación resulta especialmente adecuada para tareas que requieren una visión global del sistema y un mayor nivel de detalle en la información presentada.

Asimismo, la aplicación integra todas las funcionalidades de gestión y administración y está destinada principalmente a usuarios con privilegios elevados (SysAdmin, Gestor Almacén, Administración). La gestión de roles y permisos se detalla en la sección \ref{subsec:usuarios}. La aplicación de escritorio también puede ser usada por operarios en caso de ser necesario.

La combinación de aplicaciones de escritorio y móviles permite adaptar la interacción con el sistema a distintos contextos de trabajo, teniendo en cuenta que los perfiles de operario y gestor de almacén presentan necesidades diferenciadas. Esta aproximación contribuye a una experiencia de uso más ajustada y a un desarrollo más ordenado de la actividad dentro del almacén.

\subsection{Comunicación entre aplicaciones (APIs REST)}
\label{subsec:apis-rest}

La comunicación entre los distintos componentes del sistema puede realizarse mediante interfaces de programación de aplicaciones (APIs). Una \gls{api} puede definirse como un \footcitepage{ [...] conjunto de comandos, funciones y protocolos informáticos que permiten crear programas que interactúen con otras aplicaciones [...]}{aeb_apis}{1}. Este mecanismo resulta especialmente adecuado en escenarios donde existen clientes heterogéneos que requieren acceso controlado a la información del sistema.


En este contexto, las APIs de tipo \gls{rest} constituyen uno de los enfoques más utilizados para la comunicación entre sistemas distribuidos. \gls{rest} es un conjunto de principios arquitectónicos orientados al diseño de interfaces entre sistemas, basado en el uso del protocolo \gls{http}, o bien \gls{https} como medio de comunicación, sin añadir capas adicionales de abstracción \footbibliographypage{redes_api_rest}{10}.

Entre las principales características de las arquitecturas \gls{rest} se encuentra el empleo de \gls{http} para la realización de operaciones sobre los recursos del sistema, así como el intercambio de datos mediante formatos estructurados, siendo \gls{json} y \gls{xml} los más habituales. Los sistemas que implementan estos principios se denominan sistemas RESTful.

Una API \gls{rest}, o \gls{api} RESTful, es por tanto una interfaz que expone los recursos de un sistema siguiendo los principios \gls{rest}, permitiendo que los clientes accedan a dichos recursos mediante peticiones \gls{http} estandarizadas y reciban las respuestas en formatos estructurados. Este enfoque favorece la interoperabilidad, el desacoplamiento entre clientes y servidor y la evolución independiente de los distintos componentes del sistema.

En entornos industriales y logísticos, el uso de APIs REST permite integrar aplicaciones de escritorio, aplicaciones móviles y sistemas externos, garantizando un acceso coherente y controlado a la información del sistema. Además, este tipo de interfaces facilita la escalabilidad de la aplicación y su posible integración futura con otros sistemas de información o plataformas de automatización industrial.



\begin{figure}[H]
    \centering
    \includegraphics[width=0.7\textwidth]{Imagenes/api.png}
    \caption{Comunicación mediante APIs REST}
    \label{fig:cliente-servidor}
\end{figure}



En el sistema desarrollado, el uso de una API \gls{rest} se aplica específicamente en la comunicación entre la aplicación móvil Android y el backend del sistema, implementado mediante Spring Boot. La aplicación de escritorio, por su parte, se conecta directamente al servidor de base de datos, actuando como un cliente especializado para tareas de gestión y administración. Esta aproximación permite adaptar el mecanismo de comunicación a las necesidades y características de cada tipo de cliente.



\subsection{Backend y framework Spring Boot}
\label{subsec:spring-boot}

Spring Boot es un \gls{framework} de desarrollo basado en el lenguaje Java que facilita la creación de aplicaciones de tipo \gls{backend} y servicios web. Está diseñado para simplificar la configuración y el despliegue de aplicaciones, proporcionando una estructura predefinida y mecanismos de autoconfiguración que reducen la necesidad de configuraciones manuales \cite{aws_framework,oracle_backend,spring_boot_reference}.

En el contexto de sistemas distribuidos, Spring Boot se utiliza habitualmente para implementar \glspl{api} de tipo \gls{rest}, permitiendo exponer la lógica de negocio y el acceso a los datos mediante servicios accesibles por distintos clientes. Su integración con el ecosistema Java y su compatibilidad con tecnologías de persistencia lo convierten en una opción adecuada para el desarrollo de \glspl{backend} en entornos industriales y empresariales.



\subsection{Acceso a la base de datos en la aplicación de escritorio}
\label{subsec:desktop-db-access}

En el sistema desarrollado se ha adoptado una estrategia diferenciada para el acceso a los datos en función del tipo de aplicación cliente. Mientras que la aplicación móvil Android accede a la información del sistema a través de una \gls{api} de tipo \gls{rest}, la aplicación de escritorio establece una conexión directa con el servidor de base de datos.

Esta decisión se fundamenta en el contexto de uso de la aplicación de escritorio, concebida como una herramienta de gestión interna destinada a tareas de administración, planificación y supervisión del almacén. Dicha aplicación se ejecuta en ordenadores de escritorio dentro de la red interna de la instalación y es utilizada por perfiles de usuario con permisos elevados (SysAdmin, Gestor Almacén, Administración), lo que permite asumir un entorno controlado desde el punto de vista de seguridad y acceso.

El acceso directo a la base de datos, implementado mediante tecnologías estándar de conectividad en Java, \gls{jdbc}, permite simplificar la arquitectura del sistema para este tipo de cliente y reducir la latencia asociada a la comunicación a través de servicios intermedios.

No obstante, esta arquitectura implica un mayor grado de acoplamiento entre la aplicación de escritorio y el modelo de datos, lo que puede limitar la flexibilidad ante modificaciones en el esquema de la base de datos. Por este motivo, el acceso directo se ha restringido exclusivamente a la aplicación de escritorio y a un entorno controlado, evitando su uso en clientes móviles o en escenarios con mayores requisitos de interoperabilidad.

En conjunto, la implementación de ambos mecanismos de acceso a los datos, conexión directa a la base de datos para la aplicación de escritorio y \gls{api} \gls{rest} para la aplicación móvil, permite adaptar la arquitectura del sistema a las necesidades específicas de cada tipo de cliente. Esta estrategia, combinada con la gestión de roles y permisos, contribuye a reforzar la integridad, la eficiencia y la coherencia global del sistema.


\begin{figure}[H]
    \centering
    \includegraphics[width=0.4\textwidth]{Imagenes/interaccion-base-datos-desktop.png}
    \caption{Comunicación directa entre la aplicación de escritorio y la base de datos}
    \label{fig:cliente-servidor}
\end{figure}


% 4 Análisis del sistema
\section{Análisis del sistema}
\label{sec:analisis}

\subsection{Descripción general del sistema}
\label{subsec:descripcion-general}

El sistema desarrollado tiene como objetivo servir de herramienta de apoyo para la gestión de un almacén de consolidación en un entorno industrial. La aplicación permite centralizar la información relacionada con productos, palets, ubicaciones, movimientos y pedidos, facilitando la operativa diaria del almacén y el seguimiento de la mercancía a lo largo de los distintos procesos internos.

La solución se concibe como un sistema distribuido que da soporte a distintos perfiles de usuario mediante interfaces diferenciadas. Por un lado, una aplicación de escritorio orientada a tareas de gestión, planificación y supervisión; por otro, una aplicación móvil destinada a la operativa en planta, permitiendo el acceso a la información y el registro de acciones directamente en el entorno de trabajo.

El sistema está específicamente diseñado para almacenes de consolidación, donde se agrupan mercancías procedentes de distintos proveedores con el fin de preparar envíos conjuntos a clientes. En consecuencia, el alcance funcional se centra en procesos como la gestión de inventario, la consolidación de pedidos y la trazabilidad de los movimientos internos, sin abordar otros tipos de almacén ni el control directo de sistemas automatizados.


\subsection{Identificación de usuarios}
\label{subsec:usuarios}

\begin{table}[H]
\centering
\small
\begin{tabular}{p{6.2cm}cccc}
\toprule
\textbf{Funcionalidad} & \textbf{SysAdmin} & \textbf{Gestor Almacén} & \textbf{Operario} & \textbf{Administración} \\
\midrule
Acceso a la gestión general del almacén & \permits & \permits & \permits & \denies \\
Consulta y gestión del inventario & \permits & \permits & \permits & \denies \\
Gestión de pedidos & \permits & \permits & \denies & \permits \\
Paletización de mercancía & \permits & \permits & \permits & \denies \\
Gestión de envíos & \permits & \permits & \permits & \denies \\
Consulta del calendario de operaciones & \permits & \permits & \permits & \permits \\
Registro de movimientos de mercancía & \permits & \permits & \permits & \denies \\
Actualización de información de palets & \permits & \permits & \denies & \denies \\
Gestión de órdenes de compra & \permits & \permits & \denies & \permits \\
Exportación de datos del sistema & \permits & \denies & \denies & \permits \\
Creación de pedidos & \permits & \permits & \denies & \permits \\
Edición de pedidos & \permits & \permits & \denies & \permits \\
Eliminación de pedidos & \permits & \permits & \denies & \permits \\
Creación de productos & \permits & \permits & \denies & \denies \\
Creación de tipos de producto & \permits & \permits & \denies & \denies \\
Creación de usuarios & \permits & \denies & \denies & \permits \\
Edición de usuarios & \permits & \denies & \denies & \permits \\
Eliminación de usuarios & \permits & \denies & \denies & \permits \\
\bottomrule
\end{tabular}
\caption{Permisos de acceso a funcionalidades del sistema según rol de usuario.}
\label{tab:permisos-funcionalidades-rol}
\end{table}

El sistema contempla distintos tipos de usuarios o roles, definidos en función de su rol dentro del almacén. La aplicación de los roles se aplica exclusivamente a la aplicación de escritorio, donde se concentran las funcionalidades de gestión, administración y supervisión del sistema. En la aplicación móvil Android, orientada a la operativa en planta, todos los usuarios actúan bajo un perfil funcional equivalente al de operario, independientemente de su rol dentro de la organización.

En la aplicación de escritorio se han definido cuatro roles principales: Administrador del sistema (SysAdmin), Gestor de almacén, Operario y Administración. Cada uno de estos roles dispone de un conjunto específico de permisos que determinan el acceso a las distintas funcionalidades del sistema. La Tabla~\ref{tab:permisos-funcionalidades-rol} recoge de forma resumida los permisos asignados a cada rol en relación con las funcionalidades disponibles. Por ejemplo: Un SysAdmin puede crear, editar y eliminar usuarios, mientras que un Operario no tiene acceso a estas funciones.

El rol de SysAdmin dispone de acceso completo a todas las funcionalidades del sistema, incluyendo la gestión de usuarios, la configuración general, la administración de productos, pedidos y la exportación de datos. Este perfil está orientado a tareas de administración avanzada y mantenimiento del sistema.

El Gestor de almacén es responsable de la planificación y supervisión de la operativa diaria. Este rol puede gestionar inventario, pedidos, palets, envíos y órdenes de compra, así como realizar tareas relacionadas con la organización del almacén, pero no dispone de permisos para la gestión de usuarios ni para la exportación de datos del sistema.

El rol de Administración está orientado a tareas de carácter administrativo y documental. Este perfil puede gestionar pedidos, órdenes de compra, usuarios y exportar información del sistema, pero no interviene directamente en la operativa física del almacén ni en la gestión de movimientos de mercancía.

Por último, el Operario de almacén se encarga de ejecutar las tareas operativas relacionadas con la manipulación y movimiento de mercancías. En la aplicación de escritorio, su acceso se limita a funcionalidades operativas como la consulta de información, la paletización, el registro de movimientos y la gestión de envíos, sin permisos para tareas de administración o gestión avanzada. En la aplicación móvil Android, este rol constituye el único perfil de uso, permitiendo registrar movimientos, consultar inventario y apoyar los procesos de consolidación directamente en planta.

Esta estructura de roles y permisos permite limitar las funcionalidades y las interfaces del sistema a las responsabilidades de cada tipo de usuario, mejorando la eficiencia operativa, reforzando el control de acceso y reduciendo la probabilidad de errores derivados del uso indebido de funcionalidades no autorizadas.


\subsection{Requisitos del sistema}
\label{subsec:requisitos}


\subsubsection{Requisitos funcionales}
\label{subsubsec:requisitos-funcionales}

El sistema debe cumplir los siguientes requisitos funcionales:

\begin{itemize}
    \item Permitir la gestión de productos y sus características asociadas.
    \item Gestionar palets y su ubicación dentro del almacén.
    \item Registrar movimientos internos de mercancía y cambios de ubicación.
    \item Facilitar la preparación y consolidación de pedidos.
    \item Permitir el acceso al sistema a distintos perfiles de usuario.
    \item Proporcionar interfaces diferenciadas para tareas de gestión y operativa en planta.
    \item Consultar el estado del inventario y la trazabilidad de los productos.
\end{itemize}


\subsubsection{Requisitos no funcionales}
\label{subsubsec:requisitos-no-funcionales}

Además de los requisitos funcionales, el sistema debe cumplir una serie de requisitos no funcionales:

\begin{itemize}
    \item El sistema debe ser fácil de usar por personal con distintos niveles de experiencia.
    \item La información debe mantenerse consistente y actualizada en todo momento.
    \item El acceso a los datos debe estar controlado según el perfil de usuario.
    \item El sistema debe ser mantenible y permitir futuras ampliaciones funcionales.
    \item La solución debe ser compatible con entornos industriales habituales.
    \item El sistema debe garantizar la seguridad de las credenciales de los usuarios, evitando el almacenamiento de contraseñas en texto plano y aplicando mecanismos de protección adecuados frente a accesos no autorizados.

\end{itemize}


\subsection{Restricciones técnicas}
\label{subsec:restricciones}

El desarrollo del sistema ha estado condicionado por una serie de restricciones técnicas y de alcance propias de un Trabajo Fin de Grado.

En primer lugar, la aplicación no contempla el control directo de maquinaria ni de sistemas de automatización industrial, centrándose exclusivamente en la gestión de la información asociada al almacén. Asimismo, el sistema ha sido diseñado específicamente para un entorno de almacén de consolidación, por lo que no resulta aplicable de forma directa a otros tipos de almacén con procesos logísticos diferentes.

Por último, se han considerado las limitaciones temporales y de recursos inherentes a un Trabajo Fin de Grado, desarrollado por un único autor y dentro de un periodo de tiempo acotado. Como consecuencia, se ha priorizado la implementación de las funcionalidades esenciales para la gestión del almacén de consolidación, quedando fuera del alcance del proyecto aspectos como la integración con sistemas externos, el control directo de equipos automatizados o el desarrollo de funcionalidades avanzadas de optimización.





% 5 Diseño del sistema



\section{Diseño del sistema}
\label{sec:diseno}

\subsection{Arquitectura general del sistema}
\label{subsec:arquitectura-general}
% TODO: Explica capas, componentes, despliegue conceptual, etc.


\subsection{Diseño de la aplicación Desktop}
\label{subsec:diseno-desktop}

\begin{figure}[H]
\centering
  \includegraphics[width=1\textwidth]{Imagenes/diagrama-navegacion-ventanas.png}
  \caption{Diagrama de navegación entre ventanas de la aplicación Desktop}
  \label{fig:arquitectura-general}
\end{figure}

\begin{figure}[H]
\centering
  \includegraphics[width=1\textwidth]{Imagenes/ejemplo-navegacion-ventanas.png}
  \caption{Ejemplo de navegación entre ventanas de la aplicación Desktop (desde \textit{Login} hasta \textit{Orden Compras})}
  \label{fig:arquitectura-general}
\end{figure}

% TODO: Arquitectura interna, UI, módulos, patrones (MVC/MVVM), etc.

\subsection{Diseño de la aplicación Android}
\label{subsec:diseno-android}
% TODO: Arquitectura, navegación, pantallas, persistencia, etc.

\subsection{Diseño de la base de datos}
\label{subsec:diseno-bd}
% TODO: Modelo entidad-relación, tablas, claves, índices, restricciones.

\subsection{Diseño de la API}
\label{subsec:diseno-api}
% TODO: Endpoints, contratos, seguridad, paginación, códigos de estado.

\subsection{Diagramas UML}
\label{subsec:uml}

\subsubsection{Diagrama de clases}
\label{subsubsec:uml-clases}
% TODO: Inserta figura UML de clases.
% \begin{figure}[H]
%   \centering
%   \includegraphics[width=\textwidth]{ruta/diagrama_clases.png}
%   \caption{Diagrama de clases}
%   \label{fig:uml-clases}
% \end{figure}

\subsubsection{Diagrama de secuencia}
\label{subsubsec:uml-secuencia}
% TODO: Inserta figura UML de secuencia.

\subsubsection{Diagrama de despliegue}
\label{subsubsec:uml-despliegue}
% TODO: Inserta figura UML de despliegue.

\subsection{Diseño de la interfaz de usuario}
\label{subsec:diseno-ui}
% TODO: Mockups, guías de estilo, flujo de pantallas y decisiones UX.


% 6 Implementación
\section{Implementación}
\label{sec:implementacion}

\subsection{Herramientas y tecnologías utilizadas}
\label{subsec:herramientas}

El sistema desarrollado se ha implementado íntegramente utilizando el lenguaje de programación Java. Este lenguaje se ha empleado tanto en el desarrollo de la aplicación de escritorio como en la aplicación móvil Android y en el backend del sistema, implementado mediante el framework Spring Boot.

La utilización de un lenguaje común en todos los componentes de la solución facilita la integración entre las distintas aplicaciones, simplifica el mantenimiento del sistema y permite reutilizar conocimientos y conceptos a lo largo de todo el desarrollo. Además, Java es un lenguaje ampliamente utilizado en entornos industriales y empresariales, caracterizado por su portabilidad, robustez y amplio ecosistema de herramientas y librerías \cite{oracle_java}.


\begin{itemize}
    \item \textbf{Control de versiones}: Git.
    \item \textbf{Backend}: Java junto con el framework Spring Boot.
    \item \textbf{Base de datos}: MySQL.
    \item \textbf{Gestión y modelado de datos}: MySQL Workbench y Draw.io.
    \item \textbf{Desarrollo de la aplicación de escritorio}: IntelliJ IDEA Ultimate.
    \item \textbf{Desarrollo de la aplicación móvil}: Android Studio.
    \item \textbf{Pruebas de la API}: Postman.
    \item \textbf{Entorno de despliegue}: VMware Workstation con un servidor Ubuntu.
    \item \textbf{Gestión del proyecto}: Git.
\end{itemize}


\subsection{Implementación de la aplicación Desktop}
\label{subsec:impl-desktop}
% TODO: Estructura del proyecto, módulos, pantallas principales, etc.

\subsection{Implementación de la aplicación Android}
\label{subsec:impl-android}
% TODO: Componentes, pantallas, consumo API, gestión de estado, etc.

\subsection{Implementación del backend / servidor}
\label{subsec:impl-backend}
% TODO: Controladores, servicios, repositorios, validaciones, etc.

\subsection{Implementación de la base de datos}
\label{subsec:impl-bd}
% TODO: Scripts, migraciones, constraints, índices, datos de prueba.

\subsection{Seguridad y gestión de errores}
\label{subsec:seguridad-errores}

En el diseño del sistema se ha tenido en cuenta la protección de las credenciales de los usuarios. Las contraseñas no se almacenan en texto claro en la base de datos, sino que se guardan en forma de valores hash generados mediante un algoritmo de derivación de claves seguro.

Para este propósito se ha optado por el uso de Argon2, un algoritmo específicamente diseñado para el almacenamiento seguro de contraseñas, que ofrece resistencia frente a ataques de fuerza bruta y ataques mediante hardware especializado.


\subsection{Control de versiones y gestión del proyecto}
\label{subsec:control-versiones}
% TODO: Git, ramas, issues, milestones, metodología (Scrum/Kanban), etc.


% 7 Resultados y discusión
\section{Resultados y discusión}
\label{sec:resultados}

\subsection{Resultados del desarrollo}
\label{subsec:resultados-desarrollo}

El desarrollo del sistema ha permitido materializar una solución funcional orientada a la gestión de un almacén de consolidación en un entorno industrial, dando respuesta directa a las necesidades operativas descritas en el capítulo de introducción. El sistema resultante integra una aplicación de escritorio, una aplicación móvil para dispositivos Android y un backend este último, todos ellos conectados mediante una base de datos relacional centralizada.

La aplicación de escritorio cubre las tareas de planificación, control y gestión estructurada del almacén. En ella se han implementado funcionalidades para la gestión de productos, palets, ubicaciones, pedidos y órdenes de compra. Esta aplicación actúa como herramienta principal del sistema, facilita una visión global y actualizada del estado del inventario y de los procesos de consolidación.

La aplicación móvil se orienta a la operativa en planta, permitiendo a los operarios consultar pedidos y actualizar cantidades de producto de forma inmediata durante la ejecución de tareas físicas. Este enfoque reduce el uso de registros manuales y minimiza el desfase entre la operación real y su reflejo en el sistema, uno de los problemas identificados en el análisis inicial.

El backend implementado en la aplicación móvil centraliza la lógica de negocio y el acceso a datos, garantizando la coherencia de la información compartida por ambas aplicaciones. Las validaciones implementadas, junto con las restricciones definidas a nivel de base de datos, evitan situaciones inconsistentes como la ocupación simultánea de una ubicación por varios palets o la modificación de pedidos en estados no permitidos.

En conjunto, el sistema desarrollado permite cubrir el flujo completo de trabajo de un almacén de consolidación, desde la entrada de mercancía y su ubicación física hasta la preparación y expedición de pedidos, manteniendo la trazabilidad de palets, ubicaciones, movimientos y usuarios.

\subsection{Evaluación del cumplimiento de objetivos}
\label{subsec:evaluacion-objetivos}

Los resultados obtenidos permiten afirmar que el objetivo general del proyecto, consistente en desarrollar un sistema de gestión orientado a almacenes de consolidación en entornos industriales, ha sido alcanzado de forma satisfactoria.

En relación con los objetivos específicos definidos en el apartado~\ref{subsec:objetivos}, el sistema desarrollado responde a las necesidades operativas identificadas mediante la digitalización de los procesos clave del almacén. La arquitectura adoptada, basada en la comunicación entre aplicaciones cliente y un backend común, permite separar claramente la operativa en planta de las tareas de planificación y control, cumpliendo el objetivo de adaptar la solución a distintos perfiles de usuario.

La gestión de productos, palets, ubicaciones y pedidos se ha implementado de forma estructurada mediante un modelo de datos relacional coherente, reforzado por validaciones tanto en la lógica de negocio como en la base de datos. Este enfoque contribuye a mejorar la trazabilidad y el control del inventario, aspectos críticos señalados en la definición del problema.

Asimismo, el diseño de interfaces diferenciadas para escritorio y dispositivos móviles facilita la integración del sistema en la rutina diaria del almacén, reduciendo la complejidad percibida por el usuario final y favoreciendo su adopción en un entorno industrial real.


\subsection{Limitaciones detectadas}
\label{subsec:limitaciones}

A pesar de los resultados obtenidos, el sistema presenta limitaciones derivadas del alcance definido para el proyecto. En primer lugar, la solución se centra exclusivamente en el apoyo a la gestión del almacén, sin intervenir en el control directo de maquinaria o sistemas automatizados, tal como se estableció en la definición del problema.

La operativa está pensada para un entorno de red interna controlada, por lo que aspectos como la alta disponibilidad, el acceso remoto o la escalabilidad a gran escala no se han abordado en profundidad. Del mismo modo, la aplicación móvil cubre las tareas operativas esenciales, pero no sustituye completamente a la aplicación de escritorio en labores de planificación o gestión avanzada.



% 8 Conclusiones y trabajos futuros
\section{Conclusiones y trabajos futuros}
\label{sec:conclusiones}

\subsection{Conclusiones}
\label{subsec:conclusiones}

El presente Trabajo Fin de Grado ha abordado el diseño y la implementación de un sistema software orientado a la gestión de almacenes de consolidación en entornos industriales. Partiendo del análisis de las necesidades operativas propias de este tipo de instalaciones, se ha desarrollado una solución técnica compuesta por una aplicación de escritorio, una aplicación móvil para dispositivos Android y un backend común encargado del tratamiento y persistencia de la información.

El sistema desarrollado permite gestionar de forma estructurada los elementos fundamentales del almacén —productos, palets, ubicaciones, movimientos y pedidos— proporcionando trazabilidad completa de las operaciones realizadas y reduciendo la dependencia de procedimientos manuales o herramientas genéricas no adaptadas a este contexto. La centralización de la información en una base de datos relacional, junto con las validaciones implementadas tanto a nivel de lógica de negocio como de esquema, contribuye a mejorar la coherencia y fiabilidad de los datos.

Desde el punto de vista arquitectónico, la separación entre planificación y operativa, materializada mediante la combinación de una aplicación de escritorio y una aplicación móvil, se ajusta a la realidad de los entornos industriales analizados. Este enfoque facilita que cada perfil de usuario disponga de una interfaz adaptada a su contexto de uso, mejorando la eficiencia durante la ejecución de tareas diarias en el almacén.

En el ámbito académico, el desarrollo del proyecto ha permitido aplicar de forma integrada conocimientos relacionados con arquitectura de software, diseño de bases de datos, desarrollo de interfaces gráficas y comunicación entre sistemas distribuidos. Asimismo, ha supuesto una aproximación práctica a problemas reales de gestión logística, reforzando la capacidad de análisis, diseño y toma de decisiones técnicas propias de la ingeniería industrial.

En conjunto, los objetivos planteados al inicio del proyecto han sido alcanzados, obteniendo un sistema funcional y coherente que puede servir como base para su aplicación o ampliación en entornos reales de almacenes de consolidación.

\subsection{Líneas de mejora y trabajos futuros}
\label{subsec:trabajos-futuros}

A partir del trabajo realizado, se identifican diversas líneas de mejora y posibles extensiones que permitirían ampliar las capacidades del sistema y adaptarlo a escenarios más exigentes. Una primera línea de evolución consistiría en la integración con sistemas de automatización industrial o dispositivos de captura automática de datos, como lectores de códigos de barras, lo que permitiría reducir aún más la intervención manual, como la búsqueda por el codigo de identificación del palets, durante la operativa en planta.

Otra posible mejora se centra en la ampliación de la aplicación móvil, incorporando funcionalidades avanzadas de planificación o supervisión que actualmente se concentran en la aplicación de escritorio. Asimismo, podría evaluarse el desarrollo de una interfaz web adicional, dado que tenemos un backend implementado, que facilitase el acceso al sistema desde distintos dispositivos sin necesidad de instalación específica donde se puedan ejecutar operaciones de consulta u otras que convenga implementar.

En cuanto a la base de datos, futuras ampliaciones podrían incluir optimizaciones adicionales mediante particionado, replicación o análisis histórico de datos para la obtención de indicadores de rendimiento del almacén. Estas mejoras resultarían especialmente relevantes en instalaciones con un volumen elevado de movimientos o una rotación intensa de mercancía.

Finalmente, el sistema desarrollado puede considerarse una base sólida sobre la que continuar investigando y desarrollando soluciones software aplicadas a la gestión logística industrial, manteniendo la orientación práctica y el enfoque ingenieril que ha guiado este Trabajo Fin de Grado.

% 9 anexos
\section{Anexos}
\subsection{Script SQL completo de la base de datos}
\label{anexo:sql-esquema}

\lstinputlisting[language=SQL]{Codigo/esquema_BDD_sin_comentarios.sql} 

% =========================================================
% BIBLIOGRAFÍA
% =========================================================
\clearpage

\newgeometry{
 left=30mm,
 right=15mm,
 top=25mm,
 bottom=36mm,
 headheight=31.1pt
 }
 \setlength{\headsep}{35pt}
 %Cabecera y pie de página
\tocloftpagestyle{fancy}
\pagestyle{fancy}
\fancyhf{}
\fancyhead[LH]{\fontsize{14pt}{12} \selectfont\textsc{Diseño de una aplicación informática para la gestión de un almacén sin automatización con distintas interfaces para los operarios} \\[6pt]  \textsc{José Tomás Torre Pedroarena}}
\cfoot{\thepage}

%formato contido
\renewcommand{\cftsecfont}{\normalfont}
\renewcommand{\cftsecdotsep}{1}
\renewcommand{\cftdotsep}{1}
\renewcommand{\cftsecpagefont}{\normalfont}
\renewcommand{\cfttoctitlefont}{\fontsize{17.5pt}{0}\scshape\bfseries\color{AzulTFG}}
\renewcommand{\cftdot}{\normalfont.}
\setlength{\cftsecindent}{2em}
\setlength{\cftsubsecindent}{3em}
\setlength{\cftsubsubsecindent}{4em}
\renewcommand{\numberline}[1]{#1 \hspace{1pt}}

%formato lista figuras
\renewcommand{\cftloftitlefont}{\fontsize{17.5pt}{0}\scshape\bfseries\color{AzulTFG}}

%formato lista de tablas
\renewcommand{\cftlottitlefont}{\fontsize{17.5pt}{0}\scshape\bfseries\color{AzulTFG}}


\printbibliography[title={Bibliografía}]

\clearpage
\listoffigures
\clearpage
\listoftables


\end{document}
